\documentclass[12pt]{article}

\usepackage{graphicx}
\usepackage{paralist}
\usepackage{amsfonts}
\usepackage{amsmath}
\usepackage{hhline}
\usepackage{booktabs}
\usepackage{multirow}
\usepackage{multicol}

\oddsidemargin 0mm
\evensidemargin 0mm
\textwidth 160mm
\textheight 200mm
\renewcommand\baselinestretch{1.0}

\pagestyle {plain}
\pagenumbering{arabic}

\newcounter{stepnum}

%% Comments

\usepackage{color}

\newif\ifcomments\commentstrue

\ifcomments
\newcommand{\authornote}[3]{\textcolor{#1}{[#3 ---#2]}}
\newcommand{\todo}[1]{\textcolor{red}{[TODO: #1]}}
\else
\newcommand{\authornote}[3]{}
\newcommand{\todo}[1]{}
\fi

\newcommand{\wss}[1]{\authornote{blue}{SS}{#1}}

\title{Assignment 4,  Specification}
\author{SFWR ENG 2AA4}

\begin {document}

\maketitle

The purpose of this software design exercise is to design and implement a
portion of the specification for the FreeCell Solitaire.  This
document shows the complete specification, which will be the basis for your
implementation and testing.  

\newpage

\section* {Card ADT Module}

\subsection*{Template Module}

CardT

\subsection* {Uses}

N/A

\subsection* {Syntax}
\subsubsection* {Exported Types}
CardT = ?

\subsubsection* {Exported Access Programs}

\begin{tabular}{| l | l | l | l |}
\hline
\textbf{Routine name} & \textbf{In} & \textbf{Out} & \textbf{Exceptions}\\
\hline
CardT & string, string & CardT & \\
\hline
rank & ~ & string & ~\\
\hline
suit & ~ & string & ~\\
\hline
getCard & ~ & string & ~\\
\hline
\end{tabular}

\subsection* {Semantics}

\subsubsection* {State Variables}

r: string\\
s: string

\subsubsection* {State Invariant}

None

\subsubsection* {Assumptions}

The constructor CardT is called for each object instance before any other
access routine is called for that object.  The constructor cannot be called on
an existing object.

\subsubsection* {Access Routine Semantics}

CardT($r, s$):
\begin{itemize}
\item transition: $r, s := rank, suit$
\item output: $out := \mathit{self}$
\item exception: None
\end{itemize}

\noindent rank():
\begin{itemize}
\item output: $out := r$
\item exception: None
\end{itemize}

\noindent suit():
\begin{itemize}
\item output: $out := s$
\item exception: None
\end{itemize}

\noindent getCard():
\begin{itemize}
\item output: $out := rank() + ``\,"+ suit()$
\item exception: None
\end{itemize}


\newpage

\section* {Deck ADT Module}

\subsection*{Template Module}

DeckT

\subsection* {Uses}

CardT

\subsection* {Syntax}

\subsubsection* {Exported Types}

DeckT = ?

\subsubsection* {Exported Access Programs}

\begin{tabular}{| l | l | l | l |}
\hline
\textbf{Routine name} & \textbf{In} & \textbf{Out} & \textbf{Exceptions}\\
\hline
DeckT & ~ & DeckT & invalid\_argument\\
\hline
shuffle & ~ & ~ & ~\\
\hline
get & $\mathbb{N}$ & string &  invalid\_argument\\
\hline
\end{tabular}

\subsubsection* {State Variables}

deck: sequence of CardT

\subsubsection* {State Invariant}

$MAX\_CARD = 52$

\subsubsection* {Assumptions}

The constructor CardT is called for each object instance before any other
access routine is called for that object.  The constructor cannot be called on
an existing object.


\subsubsection* {Access Routine Semantics}

DeckT():
\begin{itemize}
\item transition: \{$i : \mathbb{N} | i \in [0..MAX\_CARDS] : deck[i] = CardT(rank[i], suit[i]) $\}\\
\item output: $out := \mathit{self}$
\item exception: exc:= $MAX\_CARDS > 52 \implies invalid\_argument$
\end{itemize}

\noindent shuffle():
\begin{itemize}
\item transition: $i : \mathbb{N} | i \in [0..MAX\_CARDS] : deck[i], deck[i+1] = deck[i+1], deck[i]$
\item exception:  None
\end{itemize}

\noindent get(i):
\begin{itemize}
\item output: $out := deck[i]$
\item exception: exc:= $i > MAX\_CARD \implies invalid\_argument$
\end{itemize}

\newpage

\section* {Board ADT Module}

\subsection*{Template Module}

BoardT

\subsection* {Uses}

CardT, DeckT

\subsection* {Syntax}

\subsubsection* {Exported Types}

BoardT = ?

\subsubsection* {Exported Access Programs}

\begin{tabular}{| l | l | l | l |}
\hline
\textbf{Routine name} & \textbf{In} & \textbf{Out} & \textbf{Exceptions}\\
\hline
BoardT & cellSize & ~ & invalid\_argument\\
\hline
dealGameCards & ~ & ~ & ~\\
\hline
moveCardTtoT & $\mathbb{N}, \mathbb{N}$ & ~ & invalid\_move\\
\hline
moveCardTtoFree & $\mathbb{N}, \mathbb{N}$ & ~ & invalid\_move\\
\hline 
moveCardFreetoT & $\mathbb{N}, \mathbb{N}$ & ~ & invalid\_move\\
\hline 
moveCardTtoFoundation & $\mathbb{N}, \mathbb{N}$ & ~ & invalid\_move\\
\hline 
isGameOver & ~ & $\mathbb{B}$ & ~ \\
\hline 
\end{tabular}

\subsubsection* {State Variables}

gameColumns: sequence of (sequence of CardT)\\
foundationRow:  sequence of (sequence of CardT)\\
freeRow:  sequence of CardT\\

\subsubsection* {State Invariant}

$Max\_Cell\_Size = 4$\\
$Max\_Cards\_inColumn = 13$\\
$Number\_of\_Columns = 8$
\subsubsection* {Assumptions}

The constructor BoardT is called for each object instance before any other
access routine is called for that object.  The constructor cannot be called on
an existing object.

\subsubsection* {Access Routine Semantics}


BoardT(cellSize):
\begin{itemize}
\item transition: \{$i : \mathbb{N} | i \in [0...Max\_Cell\_Size] \land freeRow[i] = CardT(``\,", ``\,") $\}\\
\\($i, j : \mathbb{N} | i \in [0...Max\_Cell\_Size]  \land j \in[0...Max\_Cards\_inColumn] \land foundationRow[i][j] = CardT(``\,", ``\,") $)
\item output: $out := \mathit{self}$
\item exception: exc:= $Max\_Cell\_Size \neq 4 \implies invalid\_argument$\\
\end{itemize}

\noindent dealGameCards():
\begin{itemize}
\item transition: $\{i, j : \mathbb{N} | i \in [0...(Number\_of\_Columns)] \land j \in [0...52] : gameColumns[i][j] (
\\ \begin{tabular}{| l | l | l | l |}
\hline
\textbf{i} & \textbf{j}\\
\hline
i = 0 & deck[0...7]\\
\hline
i = 1 & deck[7...14]\\ 
\hline
i = 2 & deck[14...21]\\
\hline
i = 3 & deck[21...28]\\
\hline
i = 4 & deck[28...34]\\ 
\hline
i = 5 & deck[34...40]\\
\hline
i = 6 & deck[40...46]\\
\hline
i = 7 & deck[46...52]\\ 
\hline
\end{tabular}
)$\}

\item exception: None
\end{itemize}

\noindent moveCardTtoT(column1, column2):
\begin{itemize}
\item transition: \\$gameColumns[column2].at(|gameColumns|) := gameColumns[column1][ |gameColumns| -1]$
\item exception: exc:= $(\lnot validMove(column1, column2))$
\end{itemize}
\newpage
\noindent moveCardTtoFree(column1, column2):
\begin{itemize}
\item transition: \\$freeRow[column2]:= gameColumns[column1][ |gameColumns| -1]$
\item exception: exc:= $(\lnot validMove(column1, column2))$
\end{itemize}

\noindent moveCardFreetoT(column1, column2):
\begin{itemize}
\item transition: \\$gameColumns[column1][ |gameColumns|] := freeRow[column2]$
\item exception: exc:= $(\lnot validMove(column1, column2))$
\end{itemize}

\noindent moveCardTtoFoundation(column1, column2):
\begin{itemize}
\item transition: \\$foundationRow[column1][ |gameColumns|] := gameColumn[column2][ |gameColumns| -1]$
\item exception: exc:= $(\lnot validMove(column1, column2))$
\end{itemize}

\noindent isGameOver():
\begin{itemize}
\item output: out:=($i, j : \mathbb{N} | i \in [0...|foundationRow|]  \land j \in[0...|foundationRow[i]| ] \land 
	foundationRow[i][j] = ``K") \implies TRUE $
\item exception: exc:= None
\end{itemize}

\subsection*{Local Functions}

\noindent validMove(column1, column2):
\begin{itemize}
\item transition: \\$gameColumns[column2].at(|gameColumns|) := gameColumns[column1][ |gameColumns| -1]$
\item exception: exc:= $(\lnot validMove(column1, column2))$
\end{itemize}

\noindent isAscendingTtoT(column1, column2):
\begin{itemize}
\item output: out:= \\$assignIndex(gameColumns[column1][ |gameColumns| -1]) < \\ assignIndex(gameColumns[column2][ |gameColumns| -1]) \implies TRUE$
\item exception: exc:= None
\end{itemize}

\noindent assignIndex(CardT card):
\begin{itemize}
\item transition: \\$gameColumns[column2].at(|gameColumns|) := gameColumns[column1][ |gameColumns| -1]$
\item exception: exc:= $(\lnot validMove(column1, column2))$
\end{itemize}
\end {document}